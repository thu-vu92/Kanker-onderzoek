\documentclass[a4paper,10pt]{article}
\usepackage[utf8]{inputenc}
\usepackage[left=2.54cm, top=3.54cm, right=2.54cm, bottom=2.54cm]{geometry}
\usepackage{amsmath}
\usepackage{graphicx}
\usepackage{subfig}
\usepackage{listings}
\usepackage{hyperref}

%opening
\title{Methods used}
\author{Bram, Evgeny \& Sebastiaan}

\begin{document}
\section{Cohort bias removal}
%
We apply the following bias removal methods
%
\begin{itemize}
\item L/S adjustment: cohort based normalisation $\rightarrow$ use RexR
\item Combat: Bayesian based $\rightarrow$ use \href{http://www.bu.edu/jlab/wp-assets/ComBat/Abstract.html}{1ibrary} 
\end{itemize}

We apply the cohort bias removal to the measurement cohorts. These cohorts indicate measurement batches.
%
Arguably we have to apply the bias removal, per cohort, per phenotypical cluster, otherwise
the applicability of the cohort bias removal hinges on the degree of stratification of the phenotypes.

For the RNA expression data we perform cohort bias removal with batch wise normalisation (ComBat in future work).

For the methylation data we apply quantile normalisation (functional normalisation or BEclear in future work).

To identify the cohorts that needed batch correction we compare the distribution for each cohort 
with the distribution over all the other cohorts using FDR corrected ANOVA.

\begin{itemize}
\item distribution of the log10 of the p-values, for the FDR we use the current cohort versus the rest as the label
\item distribution of median deviation
\item distribution of mean, max, min 
\item distribution of correlation values between PCA1, PCA2, PCA3
\item plots of (PCA1, PCA2, PCA3), colored by cohort.
\end{itemize}
%

\section{Batch wise normalisation}
%
Location and scale adjustment (L/S):
\begin{equation}
\mbox{Standard}\quad \mathbf{x}^*_k= \frac{\mathbf{x}_k-\overline{\mathbf{x}}_k}{\sigma_k} + \overline{\mathbf{x}}_k,\quad \forall k\in \mathcal{C}
\end{equation}
\begin{equation}
\mbox{Quantile}\quad \mathbf{x}^*_k= \frac{\mathbf{x}_k-median{(\mathbf{x}_k)}}{IQR_k} + median{(\mathbf{x}_k)},\quad \forall k\in \mathcal{C}
\end{equation}
%

In literate this approach might be referred to as \textit{standardisation}.

\section{Future work}
\subsection{Quantile normalisation}
%
Quantile normalisation: \textbf{R} $\rightarrow$ Bioconductor $\rightarrow$ minfi package $\rightarrow$ preprocessQuantile

\subsection{ComBat}
%
A type of location and scale type adjustment method
\textbf{R} $\rightarrow$ Bioconductor $\rightarrow$ sva package $\rightarrow$ ComBat

\subsection{SVA}
%
A type of location and scale type adjustment method
\textbf{R} $\rightarrow$ Bioconductor $\rightarrow$ sva package $\rightarrow$ sva



\subsection{BEclear}

\textbf{R} $\rightarrow$ Bioconductor $\rightarrow$ BEclear package $\rightarrow$ correctBatchEffect

\subsection{Functional normalisation}

\textbf{R} $\rightarrow$ Bioconductor $\rightarrow$ minfi package $\rightarrow$ preprocessFunnorm

Most batch correction methods reduce both the cohort bias \textbf{and} the biological variance.
Also, they assume \textbf{similar stratifications} over the phenotypes and it assumes \textbf{unimodality}.

\section{Predictive factors}
%
How does methylation affect the RNA expression patterns per phenotype?




\bibliographystyle{plain}
\bibliography{methods}
\end{document}
